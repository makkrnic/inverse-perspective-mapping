\documentclass[times, utf8, zavrsni, numeric]{fer}
\usepackage{booktabs}
\usepackage{listings}
\usepackage{color}
\usepackage{xcolor}
\usepackage{enumitem, hyperref}
\usepackage{wrapfig}
\usepackage{float}
\usepackage{subcaption}
\captionsetup{singlelinecheck=on}
\captionsetup{justification=centering}
\usepackage{footnote}

\DeclareCaptionFont{white}{\color{white}}
\DeclareCaptionFormat{listing}{%
  \parbox{\textwidth}{\colorbox{gray}{\parbox{\textwidth}{#1#2#3}}\vskip-4pt}}
\captionsetup[lstlisting]{format=listing,labelfont=white,textfont=white}
\lstset{frame=lrb,xleftmargin=\fboxsep,xrightmargin=-\fboxsep,language=[LaTeX]{TeX},columns=flexible}
\renewcommand{\lstlistingname}{Kod}


\begin{document}

% TODO: Navedite broj rada.
\thesisnumber{3135}

% TODO: Navedite naslov rada.
\title{Inverzna perspektivna transformacija slike ravninskog objekta}

% TODO: Navedite vaše ime i prezime.
\author{Mak Krnic}

\maketitle

% Ispis stranice s napomenom o umetanju izvornika rada. Uklonite naredbu \izvornik ako želite izbaciti tu stranicu.
\izvornik

% Dodavanje zahvale ili prazne stranice. Ako ne želite dodati zahvalu, naredbu ostavite radi prazne stranice.
\zahvala{}

\tableofcontents

\chapter{Uvod}
\label{ch:uvod}

Projekcijsko ravninsko preslikavanje se u području računalnog vida koristi za mnoge primjene. Neke od primjena su ispravljanje izobličenja nastalih zbog perspektivne slike na kojoj su promijenjeni odnosi linija, zrcaljenje slike, "dodavanje" perspektive na sliku ili rotacija slike.

Cilj ovog rada bio je implementirati algoritam za izračunavanje matrice inverzne perspektivne transformacije ($H$) na temelju korisničkog odabira četiri točke te pomoću te matrice izračunati inverznu perspektivnu transformaciju ulazne slike. Kao ulazna datoteka može se iskoristiti slika u boji (RGB\footnote{Red Green Blue}), bilo kojih dimenzija.

Implementacija je izvedena u jeziku C++ koristeći biblioteku OpenCV nakon što je napravljen prototip u pythonu. Nakon pokretanja od korisnika se traži da odabere četiri točke, te započinje računanje transformacijske matrice i same perspektivne transformacije slike.

Svojstva i algoritam izračuna transformacijske matrice $H$ i perspektivne transformacije opisani su u poglavlju \ref{ch:preslikavanje}, Nakon toga u poglavlju \ref{ch:implementacija} slijedi programska implementacija s odsječcima kôda u Pythonu i C++-u uz pojašnjenja te popis i informacije o korištenoj programskoj podršci i bibliotekama u poglavlju \ref{ch:podrska}. U poglavlju \ref{ch:rezultati} naveden je primjer izvođenja te postignuti rezultati s usporedbom brzine izvođenja programa u C++-u i Pythonu. Naposlijetku je izveden zaključak u poglavlju \ref{ch:zakljucak}, navedena korištena literatura te sažetak najbitnijih točaka.
\chapter{Projekcijsko ravninsko preslikavanje}
\label{ch:preslikavanje}

\section{Prikaz točaka u ravnini homogenom notacijom}
\label{sec:homNot}

Svaka točka u ravnini u pravokutnom koordinatnom sustavu jednoznačno je određena uređenim parom koordinata $(x, y)$ te je stoga uobičajeno ravninu poistovjetiti s $\mathbb{R}^2$. Zbog toga se točku u ravnini može prikazati vektorom $\mathbf{x} = (x, y)^\top$.

Kao što se točka u ravnini može jednoznačno predstaviti  parom koordinata $(x,~y)$, tako se pravac može prikazati jednadžbom $ax + by + c = 0$, gdje se različiti pravci dobivaju mijenjajući parametre $a$, $b$ i $c$. Zbog tog se svojstva pravci mogu bez promjene mogu prikazati u homogenoj notaciji kao stupac-vektor $(a, b, c)^\top$. Ovaj prikaz, doduše, nije jednoznačan, budući da vektori $(a, b, c)$ i $(ka, kb, kc)$ predstavljaju isti pravac za svaki $k \neq 0$, ali za svaki za svaki pravac zapisan u homogenoj notaciji postoji točno jedan pravac zapisan u euklidskoj notaciji \citep{Hartley2004}.

Da bi se točka $\mathbf{x} = (x, y)^\top$ zapisala pomoću homogenih koordinata, potrebno je uzeti u obzir slijedeće:
\begin{enumerate}
	\item \label{itm:fst} Ako i samo ako točka $\mathbf{x} = (x, y)^\top$ leži na pravcu $\mathbf{I} =(a, b, c)^\top$, onda vrijedi jednakost $ax + by + c = 0$.
	\item \eqref{itm:fst}  se može zapisati kao skalani produkt vektora koji prikazuje točku i homogenog prikaza pravca: $(x, y, 1) \cdot (a, b, c) = (x, y, 1) \cdot \mathbf{I} = 0$.
	\item Ako i samo ako vrijedi $(x, y, 1) \cdot \mathbf{I} = 0$, onda vrijedi i  $(kx, ky, k) \cdot \mathbf{I} = 0$ za bilo koju konstantu $k \neq 0$.
\end{enumerate}

Iz gorenavedenoga vidi se da će točki $\mathbf{x} = (x_1, x_2, x_3)^\top$ iz projekcijske ravnine $\mathbb{P}^2$ odgovarati točka $(\frac{x_1}{x_3}, \frac{x_2}{x_3})^\top$ u euklidskoj ravnini $\mathbb{R}^2$, gdje je $x_3$ homogena koordinata točke $\mathbf{x}$.

Važno je još napomenuti da se u slučaju kada je homogena koordinata točke $\mathbf{x}$ jednaka $0$ kaže da je ta točka u euklidskoj ravnini u beskonačnosti.

\section{Projekcijsko preslikavanje}
\label{sec:projPresl}

Planarna projekcijska transformacija ili \textit{homografija} je linearna transformacija nad homogenim 3-dimenzionalnim vektorom, a koja se može prikazati nesingularnom $3 \times 3$ matricom \citep{Hartley2004}:
\begin{equation}
	\label{eq:transLong}
	\left[\begin{matrix}
		x'_1 \\
		x'_2 \\
		x'_3
	\end{matrix}
	\right] \sim \left[\begin{matrix}
		h_{11} & h_{12} & h_{13} \\
		h_{21} & h_{22} & h_{23} \\
		h_{31} & h_{32} & h_{33}
	\end{matrix}
	\right]
	\cdot
	\left[
	\begin{matrix}
		x_1 \\
		x_2 \\
		x_3
	\end{matrix}
	\right]
\end{equation}
ili kraće
\begin{equation}
	\mathbf{x'} \sim H \cdot \mathbf{x}
	\label{eq:transShort}
\end{equation}

Važno je još primjetiti da se projekcijska transformacija ne mijenja skaliranjem transformacijske matrice $H$ faktorom različitim od nule, pa se zbog toga matrica $H$ naziva \textit{homogenom matricom} jer je bitan samo omjer elemenata, ali ne i same njihove vrijednosti. Budući da postoji 8 međusobno nezavisnih omjera, kažemo da projekcijska transformacija ima 8 stupnjeva slobode.

\section{Interpolacija}
\label{sec:interpolacija}

Interpolacija je matematička metoda kojom se na tememlju poznatog diskretnog skupa funkcijskih vrijednosti konstruiraju nove funkcijske vrijednosti. Diskretni se skup vrijednosti dobiva eksperimentalno ili uzorkovanjem te se interpolacija koristi kao jedna od metoda za rekonstruiranje funkcije. \citep{_interpolation_2013}

% TODO: u ovom radu koristi se ...

\subsection{Linearna interpolacija}
\label{subsec:linInt}
\label{interpolation:lin}

Linearna interpolacija jedna je od najjednostavnijih metoda interpolacije namjenjen interpolaciji točaka sa samo jednom prostornom dimenzijom, odnosno za procjenu vrijednosti funkcije koja ovisi samo o jednom parametru. 

Pozunajući točke $A (x_1, y_1)$ i $B (x_2, y_2)$, linearni interpolant je dužina koja spaja te dvije točke. Pravac na kojem ta dužina leži izračunava se prema formuli za pravac kroz dvije točke \citep{_linear_2013}
\begin{equation}
	y = y_1 + (y_2 - y_1) \frac{x - x_1}{x_2 - x_1}, y_n = f(x_n)
\end{equation}

Linearna interpolacija nad skupom točaka dobiva se konkatenacijom pojedinih linearnih interpolacija između svakog para točaka, te je tada rezultat linearne interpolacije neprekinuta krivulja čija derivacija ima prekide u poznatim točkama \citep{Bosilj2010} \citep{_linear_2013}.

\subsection{Bilinearna interpolacija}
\label{subsec:bilinInt}

Bilinearna interpolacija je proširenje linearne interpolacije (\ref{interpolation:lin}), a koristi se za interpolaciju funkcije dvije varijable na pravokutnoj 2D mreži. Provodi se tako da se provede \textbf{linearna} interpolacija prvo u jednom smjeru, a zatim u drugom \citep {_bilinear_2013}. Nije bitno koji je smjer prvi.

Ako su nam poznate vrijednosti funkcije u točkama $Q_{11} (x_1, y_1)$, $Q_{12} (x_1, y_2)$, $Q_{21}  (x_2, y_1)$ i $Q_{22} (x_2, y_2)$, onda bilinearnom interpolacijom možemo naći nepoznatu vrijednost funkcije $f$ u točki $P (x, y)$ koja se nalazi unutar pravokutnika omeđenog točkama $Q_{11}$, $Q_{12}$, $Q_{21}$ i $Q_{22}$.

Na slici \eqref{fig:bilintPrikaz} se može vidjeti grafički prikaz izračuna vrijednosti interpolirane točke.

\begin{figure}[ht]
\centering
\includegraphics[width=6cm]{figures/bilint_visualization.png}
\caption{Grafički prikaz bilinearne interpolacije}
\label{fig:bilintPrikaz}
\end{figure}

Za primjer, pokazat ćemo algoritam kada se linearna interpolacija radi prvo po $x$-osi, a zatim po $y$-osi.
\begin{equation}
f(R_1) \approx \frac{x_2 - x}{x_2 - x_1} f(Q_{11}) + \frac{x - x_1}{x_2 - x_1} f(Q_{21})\text{ ,}
\end{equation}
gdje je  $R_1 = (x, y_1)$.
\begin{equation}
f(R_2) \approx \frac{x_2 - x}{x_2 - x_1} f(Q_{12}) + \frac{x - x_1}{x_2 - x_1} f(Q_{22})\text{ ,}
\end{equation}
gdje je  $R_2 = (x, y_2)$.

Zatim nastavljamo provodeći interpolaciju po $y$-smjeru:
\begin{equation}
f(P) \approx \frac{y_2 - y}{y_2 - y_1} f(R_1) + \frac{y - y_1}{y_2 - y_1} f(R_2).
\end{equation}

U ovom se radu bilinearna interpolacija koristi za određivanje boje slikovnog elementa \engl{pixel} u slučaju kada originalne koordinate transformirane točke nisu cjelobrojne, te se zbog toga može koristiti pojednostavljena verzija algoritma u kojoj se pretpostave koordinate točaka $(0,0)$, $(0, 1)$, $(1, 0)$ te $(1, 1)$:
\begin{equation}
f(x, y) \approx f(0, 0)(1 - x)(1 -  y) + f(1, 0) x (1 - y) + f(0, 1) (1 - x) y + f(1, 1) x y
\end{equation}

\section{Određivanje transformacijske matrice}
\label{sec:odredjivanjeTransMat}

Ako su nam poznate koordinate četiriju točaka na originalnoj slici $I_o$ te znamo u koje se one točke preslikavaju na transformiranoj slici $I_t$, može se riješiti jednadžba \eqref{eq:transShort} i na taj način dobiti transformacijsku matrica $H$ \citep{segvicDinAn3D}.

S obzirom na to da nam činjenica da se transformacijska matrica $H$ može množiti koeficijentom uvelike komplicira situaciju, kako bi pojednostavili jednadžbu \eqref{eq:transShort} za svaku točku, možemo je zapisati na slijedeći način:
\begin{equation}
q_t = \lambda \cdot H \cdot q_o, \lambda \in \mathbb{R}
\label{eq:transMatSimpler}
\end{equation}
gdje je $q_t$ transformirana točka, a $q_o$ originalna.

Nadalje, iz jednadžbe \eqref{eq:transMatSimpler} se može iščitati da su $q_t$ i $H \cdot q_o$ paralelni, što znači da je njihov vektorski produkt jednak $\mathbf{0}$ \citep{vecParallel}:
\begin{equation}
\label{eq:crossProd}
q_t \times (H \cdot q_o) = \vec{0}
\end{equation}
Ako transformacijsku matricu $H$ iz \eqref{eq:transLong} i \eqref{eq:transShort} radi jednostavnijeg zapisa zapišemo vektor-stupac u kojem je svaki element vektor-redak:
\begin{equation}
H = \left[
\begin{matrix}
h_{11} & h_{12} & h_{13} \\
h_{21} & h_{22} & h_{23} \\
h_{31} & h_{32} & h_{33}
\end{matrix}
\right]
= \left[
\begin{matrix}
\mathbf {h_1} \\
\mathbf{h_2} \\
\mathbf{h_3}
\end{matrix}
\right]
\end{equation}
onda dalje \eqref{eq:crossProd} postaje
\begin{equation}
\label{eq:transCrossProdLong}
\mathbf{q_t} \times \left[
\begin{matrix}
\mathbf {h_1} \cdot \mathbf{q_o} \\
\mathbf{h_2} \cdot \mathbf{q_o} \\
\mathbf{h_3} \cdot \mathbf{q_o}
\end{matrix}
\right]
= \left[
\begin{matrix}
q_{tx} \\
q_{ty} \\
q_{tz}
\end{matrix}
\right]
\times \left[
\begin{matrix}
\mathbf {h_1} \cdot \mathbf{q_o} \\
\mathbf{h_2} \cdot \mathbf{q_o} \\
\mathbf{h_3} \cdot \mathbf{q_o}
\end{matrix}
\right]
= \mathbf{0}
\end{equation}

Budući da su i transformirana ($q_t$) i originalna ($q_o$) točka zadane u euklidskoj notaciji, njihove su homogene koordinate -- $q_{tz}$ i $q_{oz}$ -- jednake $1$ te \eqref{eq:transCrossProdLong} postaje:
\begin{equation}
\label{eq:transCrossProdDetailed}
\left[
\begin{matrix}
q_{tx} \\
q_{ty} \\
1
\end{matrix}
\right]
\times \left[
\begin{matrix}
\mathbf{h_1} \cdot \mathbf{q_o} \\
\mathbf{h_2} \cdot \mathbf{q_o} \\
\mathbf{h_3} \cdot \mathbf{q_o}
\end{matrix}
\right]
= \left[
\begin{matrix}
q_{ty} \cdot \mathbf{h_3} \cdot \mathbf{q_o} - \mathbf{h_2} \cdot \mathbf{q_o} \\
-q_{tx} \cdot \mathbf{h_3} \cdot \mathbf{q_o} + \mathbf{h_1} \cdot \mathbf{q_o} \\
q_{tx} \cdot \mathbf{h_2} \cdot \mathbf{q_o} + q_{ty} \cdot \mathbf{h_1} \cdot \mathbf{q_o}
\end{matrix}
\right]
= \mathbf{0}
\end{equation}
Nakon što izlučimo $h_1$, $h_2$ i $h_3$ dobivamo:
\begin{equation}
\left[
\begin{matrix}
0 & 0 & 0 & -q_{ox} & -q_{oy} & -1 & q_{ty} q_{ox} & q_{ty} q_{oy} & q_{ty} \\
q_{ox} & q_{oy} & 1 & 0 & 0 & 0 & -q_{tx} q_{ox} & -q_{tx} q_{oy} & -q_{tx} \\
-q_{ty} q_{ox} & -q_{ty} q_{oy} & -q_{ty} & q_{tx} q_{ox} & q_{tx} q_{oy} & q_{tx} & 0 & 0 & 0
\end{matrix}
\right]
\cdot
\left[
\begin{matrix}
h_{11} \\
h_{12} \\
h_{13} \\
h_{21} \\
h_{22} \\
h_{23} \\
h_{31} \\
h_{32} \\
h_{33}
\end{matrix}
\right]
\end{equation}
ili kraće
\begin{equation}
\left[
\begin{matrix}
\mathbf{0}_3^\top & -\mathbf{q_o}^\top & q_{ty} \mathbf{q_o}^\top \\
\mathbf{q_o}^\top & \mathbf{0}_3^\top & -q_{tx} \mathbf{q_o}^\top \\
-q_{ty}\mathbf{q_o}^\top & q_{tx} \mathbf{q_o}^\top & \mathbf{0}_3^\top
\end{matrix}
\right]
\cdot
\left[
\begin{matrix}
\mathbf{h_1}^\top \\
\mathbf{h_2}^\top \\
\mathbf{h_3}^\top
\end{matrix}
\right]
= \mathbf{0}_3
\end{equation}

Za svaki par točaka dobivamo po dvije nezavisne jednadžbe, dakle, za četiri para dobivamo linearni sustav oblika:
\begin{equation}
\label{eq:transSystemShort}
A_{8\times9} \cdot \mathbf{h}_{9\times1} = \mathbf{0}_{8\times1}
\end{equation}
Uočavamo da sustav iz \eqref{eq:transSystemShort} ima 8 jednadžbi i 9 nepoznanica. Taj je sustav moguće rješiti zahvaljujući njegovom svojstvu homogenosti\footnote{Homogen sutstav je sustav oblika $A\mathbf{x} = \mathbf{0}$} metodom singularne dekompozicije \engl{SVD, singular value decomposition}.

\chapter{Programska implementacija}
\section{Prototip u pythonu}
\section{Implementacija u C++-u}
\lstset{language=C++, tabsize=2}
\begin{lstlisting}
int main () {
	return 0;
}
\end{lstlisting}
\chapter{Korištena programska podrška i biblioteke}
\label{ch:podrska}

U ovom poglavlju naveden je popis korištene programske podrške te za svaku stavku kratak opis upotrebe u okviru ovog rada.

\section{Python}
\label{sec:podrskaPython}

Python je interpretirani, objektno orjentirani viši programski jezik. Njegove ugrađene strukture i metode kombinirane s dinamičkim tipiziranjem i povezivanjem čine ga veoma popularnim jezikom za brzi razvoj aplikacija \engl{RAD -- Rapid Application Development} i brzo prototipiranje, kao i za pisanje skripti ili za povezivanje već postojećih komponenti. Python ima jednostavnu sintaksu te je fokusiran prvenstveno na čitljivost koda, čime se olakšava održavanje programa. Podržava module i pakete, čime se potiče na modularno razvijanje programa i ponovno korištenje koda. Python interpreter i standardna biblioteka su dostupni u obliku izvornog koda ili binarne datoteke besplatno i smiju se slobodno distribuirati \citep{Python}.

\subsection{Python Imaging Library (PIL)}
\label{subsec:pil}
\emph{Python Imaging Library} je vanjska biblioteka za Python koja se koristi za učitavanje, konverziju, spremanje i stvaranje novih slika \citep{PIL}

\subsection{pylab}
\label{subsec:pylab}

\emph{PyLab} je modul biblioteke Matplotlib koji služi za iscrtavanje raznih grafičkih elemenata na slikama. U okviru ovog rada koristi se metoda ginput koja vraća točke koje je korisnik odabrao pokazivačem.

\subsection{numpy}
\label{subsec:numpy}

\emph{NumPy} je paket za Python koja dodaje podršku za znastvene proračune. To je biblioteka koja pruža mogućnost rada s višedimenzionalnim nizovima i izvedenim objektima (kao npr. matricama) i sadržio funkcije za brze operacije nad nizovima \citep{NumPy}.

\section{C++}
\label{sec:c++}

\emph{C++} je niži (prema današnjim mjerilima) objektno orjentirani, strogo tipizirani, kompajlirani jezik. Zbog njegove bliskosti hardveru, a i zato što se prevodi u strojni kod u cjelosti prije izvođenja, veoma se brzo izvodi. Sadrži veoma opsežne standardne biblioteke i nebrojeno mnogo vanjskih biblioteka.

\subsection{OpenCV}
\label{subsec:opencv}

\emph{OpenCV (Open source computer vision library)}  je biblioteka otvorenog koda koja pruža mogućnost rada s računalnim vidom i strojnim učenjem. OpenCV je stvoren kako pružio zajedničku infrastukturu za rad s računalnim vidom i kako bi olakšao i ubrzao razvoj softvera koji koristi računalnu percepciju \citep{OpenCV}.
\chapter{Izvođenje i rezultati}
\label{ch:rezultati}

\section{Primjer izvođenja programa}
\label{sec:izvodjenje}

Program se pokreće iz naredbenog retka naredbom \lstinline!./inverseMapping! kao u prikazu koda \ref{code:pokretanje}.
\begin{lstlisting}[label={code:pokretanje}, caption={Pokretanje programa}]
./inverseMapping originalna_slika.jpg transformirana_slika.jpg sirina_transfomirane_slike visina_transformirane_slike
\end{lstlisting}

Nakon toga se prikazuje prozor s originalnom slikom kao na slici \ref{fig:odabirTocaka1} na kojemu korisnik pokazivačem odabere četiri točke koje predstavljaju kutove dijela slike nad kojim je potrebno obaviti inverznu perspektivnu transformaciju, počevši od lijevog gornjeg u smjeru kazaljke na satu (slika \ref{fig:odabirTocaka2}).

\begin{figure}[ht]
\centering

\begin{subfigure}{0.45\textwidth}
\begin{minipage}{0.9\textwidth}
	\centering
	\includegraphics[width=\textwidth]{figures/points_selection.jpg}
	\caption[margin=1cm]{Originalna slika prije odabira točaka}
	\label{fig:odabirTocaka1}
\end{minipage}
\end{subfigure}%
\begin{subfigure}{0.45\textwidth}
\begin{minipage}{0.9\textwidth}
	\centering
	\includegraphics[width=\textwidth]{figures/points_selection_selected.jpg}
	\caption{Originalna slika za vrijeme odabira točaka}
	\label{fig:odabirTocaka2}
\end{minipage}
\end{subfigure}

\caption{Odabir točaka}
\label{fig:odabirTocaka}

\end{figure}

Nakon odabira točaka počinje izračunavanje transformacijske matrice, a nakon toga i sama transformacija slike čiji se napredak može pratiti u konzoli. Kada su svi slikovni elementi transformirani, transformirana se slika sprema u datoteku navedenu u naredbenom retku, a istovremeno se korisniku prikaže prozor s rezultatom, tj. istom tom transformiranom slikom (slika \ref{fig:transformiranaSlika}).

\begin{figure}[ht]
	\centering
	\includegraphics[width=0.7\textwidth]{figures/transformedImage.jpg}
	\caption{Rezultantna slika nakon transformacije}
	\label{fig:transformiranaSlika}
\end{figure}

Program završava pritiskom na tipku <ESC>

\section{Rezultati}
\label{sec:rezultati}
\chapter{Zaključak}
\label{ch:zakljucak}
Zaključak.


\bibliographystyle{fer}
\bibliography{literatura}

\begin{sazetak}

U ovom radu implementiran je algoritam za izračun matrice inverzne perspektivne transformacije na temelju korisničkog unosa četiri točke. Inverzna perspektivna transformacija je važan postupak područja računalnog vida s raznim primjenama. Za potrebe izračuna matrice proučena je matematička metoda singularne dekompozicije \engl{SVD}, a za samu implementaciju biblioteka \emph{OpenCV} za C++ te biblioteke \emph{Python Imaging Library}, \emph{PyLab} i \emph{NumPy} za Python. Provedena su testiranja i usporedbe vremena izvođenja te je utvrđeno da se implementacija u C++-u izvodi višestruko brže od implementacije u Pythonu te da je implementacija u C++-u pogodna za obradu slike u stvarnom vremenu.

\kljucnerijeci{projekcijsko ravninsko preslikavanje, homografija, projekcijska transformacija, računalni vid, obrada slike, izračun matrice homografije}
\end{sazetak}

% TODO: Navedite naslov na engleskom jeziku.
\engtitle{Inverse perspective mapping of a planar object}
\begin{abstract}

This work implements the algorithm for the calculation of the inverse perspective transformation matrix based on user-selected four points. Inverse perspective mapping is an important method in the field of computer vision with various applications. For the matrix computation attention was given to the method of singular value decomposition, and for the implementation itself \emph{OpenCV} library for C++ and the libraries \emph{Python Imaging Library (PIL)}, \emph{PyLab} and \emph{NumPy} for Python. The testing was conducted noting execution times and it has been noticed that the C++ implemtation's execution time is many times shorter than that of Python's and that the C++ implementation is suitable for real-time image processing.

\keywords{projective planar mapping, homography, projective transformation, computer vision, image processing, homography matrix calculation}
\end{abstract}

\end{document}
